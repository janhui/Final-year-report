
\addcontentsline{toc}{chapter}{Abstract}

\begin{abstract}

Interactive tables that allow people to dynamically interact with their workspace is a concept that has been around for a long time. Science fiction and novel research have inspired the development of many innovative prototypes since the early 90s. However there are still notably little commercial options available  in the consumer space. Larger companies have recently realised this untapped potential.

However most of these products are still too expensive and not viable for large public sector organisations such as libraries and schools. There is a definite market for a cheaper alternative that the majority of users can utilise. One research program that is making positive progress in this territory is the HuddleLamp project.

Over the course of this thesis we have looked at alternatives to some of the technologies used by HuddleLamp, to create a more portable system. We have taken an in depth look into computer vision algorithms for mobile devices and also various storage technologies.

We have endeavoured to take the next step, in the evolution of the project, by attempting to create a system that uses radio frequency signals and trilateration algorithm. We have described and evaluated the various pitfalls we have encountered over the course of the project and the ways in which we have overcome them.

Although our attempt in using the Bluetooth Low Energy technology did not give a positive result, we were able to gain valuable insight into the underlying problems and able to identify the next step forward in this project. By eliminating one method of positioning and by creating a simple extendable data storage API using a prominent data store like Firebase, we have reduced the complexity involved in future revisions to the HuddleTable project.

\end{abstract}
