\subsection{Summary}
After careful consideration of the available options I had for each choices. I decided to go with Java as my language choice and Android as my platform. Mainly due to my familiarity with both as well as the fact that android devices have a larger market share, therefore more people are likely to have one in their possession.  I decided to use OpenCV as my vision library due to its cross platform availability and also because it worked on all hardware. It also had the added bonus of being the more mature library compared to FastCV.

I decided to use contour detection and colour detection for my main graphics manipulation and device identification, because both were less resource intensive algorithms. This meant that it was the most efficient of the choices I had due to the limited resources of the devices. Using something that worked well with a less powerful processor and smaller memory was one of the priorities of this project.

I also decided to use Firebase as the data store of my system because of its real time data synchronisation abilities. The opportunity to get real time update and to push data from the storage without having to worry about concurrency issues was too good to ignore.