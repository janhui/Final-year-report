\chapter{Device Camera}

\label{ch:devcamera}
\section{Design Choices}
\subsection{Language choice}


\subsubsection{Java}

Java is one of the most widely used language fro development. It also
has the unique advantage that Java programs compiles to a Java Bytecode,
which runs on a Java Virtual Machine. This means that as a developer
they would only usually have to create the program once and it would
run on most machines. 

Java also happens to be the native language for Android, making it
an ideal choice of language for an Android developer. Android platform
has a market share of around 81.5\%\cite{market-share}. Which means
creating a library for android made the perfect sense. More people
are likely to have an Android device during a meeting and also there
is a more likely hood that there would a lot more spare Android devices
which can be used to create the screen for the table. Android devices
are also extremely cheap which means even if the users had to buy
some devices to use the interactive table it would be relatively cheap.

By creating the library in Java there would be more likely hood for
adoption by other users due to the ease in which someone could pick
it up and also due to the large number of Android developers other
there already. There are lot of vendors that produce cheap Android
devices\cite{cheap-android}. 

One of the biggest disadvantage of using Android devices with BLE
technology is the the Android Bluetooth Crash bug\cite{bluetooth-share}.
There is a bug in Bluetooth stack of Android where an Android device
can only see 1,990 different Bluetooth MAC addresses before the Android
BluetoothService crashes. Android also periodically saves its list
of recent mac address which means this stored list could get filled
up pretty quickly. Small iBeacons such as the Gimbal by Qualcomm generates
a new MAC address for itself every 0.8 seconds which means this list
could get filled very quickly. When is this list gets filled up an
error dialog saying \textquotedbl{}Unfortunately, Bluetooth Share
has Stopped\textquotedbl{} shows up. The only real solution to this
problem is a downloading an called Bluetooth Crash Resolver by radius
network which is not a perfect solution but the best there is right
now.


\subsubsection{Object C/Swift}

Object C/ Swift is native language for iOS platform used by the Apple
devices. With a 14.8\% market share iOS is the second biggest smartphone
platform out there\cite{market-share}. They recently brought out
another language called Swift which is also another language that
could used for development in the iOS platform. iOS was a platform
that was brought in 2007 and over the last 8 years, over one billion
iOS devices has been sold\cite{oneBios}.

Over 1.4 million iOS applications has been published in Apples App
store, showing that large number of developers has adopted Apple products
and are creating Apps for the platform. Apple is also the company
the coined the term iBeacon which are a ``new class of low-powered,
low-cost transmitters that can notify nearby iOS 7 or 8 devices of
their presence.'' This means that iOS platform has a more mature
codebase for BLE technology and also there are a lot more support
for it. The iOS CoreLocation\cite{corelocation} framework is a relatively
more accurate feature giving iOS a very good advantage while working
with BLE technology. The Bluetooth stack used by Apple product is
a much more efficient one.

However for a developer there quite a lot of requirements before being
able to develop for an Apple platform. Unlike Android development
an iOS developer needs to have an Apple computer and a Apple developer
License before they can create apps for the platform, neither of which
are cheap. Apple devices are also rather expensive to buy and there
fore unlike for people to have it just laying around.


\subsubsection{JavaScript}

JavaScript is also another viable language for creating applications
for mobile devices especially since the introduction of frameworks
such as PhoneGap\cite{phonegap} and Ionic\cite{ionic}. Mobile Frameworks
such as these work on top Apache Cordova which allows developers to
create a application using web technologies such as HTML5, CSS3 and
JavaScript instead of using the platform specific APIs like those
in iOS and Android. Cordova wraps these web application to work on
different devices. These applications are called Hybrid applications
since they are not truly native mobile application or web application.

The use of web-based technologies leads many PhoneGap applications
to run slower than native applications with similar functionality.
However PhoneGap supports development for iOS, Blackberry, Android,
LG webOS, Microsoft Windows Phone (7 and 8), Nokia Symbian OS, Tizen
(SDK 2.x), Bada, Firefox OS, and Ubuntu Touch making it one of the
best language to use while creating an application to get the most
exposure for the library. it also supports most of the features of
a phone that a native language does as well like Accelerometer, Compass
etc as shown in \ref{tab:PhoneGap-table}.


\subsection{Data Storage}


\subsubsection{Firebase}

Firebase\cite{firebase} is a relatively new back-end as a service
company recently acquired by Google. They are a cloud service provider
from San Francisco who caters a number of products for software developers
building mobile or web application. The service we are more interested
is their Real-time Database service. It provides developers with an
API that which enables data to be synchronized across clients automatically
and stored on the Firebase cloud. The main advantage is the fact that
the data gets synchronized within milliseconds. It also supports a
multitude of platforms such as Android, iOS, JavaScript etc.

It also only requires client-side code meaning the developers has
a relative easy job to create the back-end they required with. Storing
the information as JSON with every information accessible through
their own URL means the data is easy to store and retrieve and there
is also a readily available rest endpoint. 

By providing a simple security management system, Firebase ensures
that the data moved through the apps are encrypted and safe. By ensuring
that the security measures are enforced across all parts of the system,
It helps the developer in eradicating glaring security risks. Firebase
built with scaling and performance in mind. When the data changes
Firebase calculates the minimum number of changes required to keep
all the clients in synch. It also scales linearly depending on the
amount of data being synchronized, decreasing the likely hood of a
developer error. 

The potential of Firebase is enormous. The creators of Firebase hope
that it can serve the needs of 95\% of web service\cite{firebase-wired},
meaning it will be an ideal tool in the stack of our system. One of
its glaring problems is the it is not ideal for processing images,
however the real-time data synchronization aspect of it make it a
really good candidate for the HuddleTable.


\subsubsection{Meteor}

Meteor\cite{meteor} is a real-time JavaScript framework to come from
the Y-Combinator incubator in 2011. It is primarily used by developers
due to its ability in rapid-prototyping, hence for a MEng project
stack, meteor is an ideal framework to use. It produces cross-platform
code such that for a multiple mobile devices project like HuddleTable
it would cut down on the work required to get most of the devices
to work. It is tightly coupled with the MongoDB, a prominent No-SQL
database. It has an in built publish\textendash subscribe pattern\cite{pub-sub-pattern}
which helps to propagate data changes to clients in real-time without
the need for synchronization protocols.

Some of the design principles used by Meteor makes it a really good
choice for HuddleTable. It has a Data on the wire motivation meaning,
it only sends the minimum data necessary to re-render the portion
of the page that has changed decreasing the latency. It has a full
stack reactivity\cite{meteor-wiki} meaning that everything gets updated
together when required encouraging its simplicity motivation. Its
created in such a way that it is easy to learn and implement even
for beginners. 

Meteor is the framework of choice used the by HuddleLamp research.
They developed an Object storage on top of Meteor so that all manipulations
on one device become instantly visible on all the other devices. This
allowed the cross platform collaborative work to be done with ease.
Having to only work with one language, JavaScript, on the whole project
means a lot less work for the developer, however the complext algorithm
required for HuddleTable would not be enticing to do in JavaScript.


\subsubsection{AirDrop / Android Beam}

AirDrop\cite{airdrop} is another data sharing service started in
2011 by Apple. It lets you transfer files, links and other content
over the Wi-Fi and Bluetooth between Apple devices. It is a technology
that has been widely appreciated by Apple users and highly congratulated
on. The simple interface to send files back and forth without any
``handshake'' like system and having no restrictions in size of
the files transferred means its a highly appreciated technology which
doesn't require internet for it to work. It also has built in function
to control the recipients of the information sent. AirDrop creates
a peer-to-peer Wi-Fi network between the devices and the files are
sent encrypted which means the file transfer is also secure.

Android Beam\cite{android-beam} is a similar technology developed
by Android for their devices. It allows android devices to share content
with each other by pressing the devices back to back. Android uses
the Near field Communication (NFC) hardware to transfer the information
between the devices. However compared to the Apple's AirDrop, Android
beam is slower due to it using NFC instead of Wi-Fi. NFC is also not
available on all Android devices.

These technologies are restricted because they would only work with
each other. By creating a protocol similar to this which would work
across platform would be an ideal scenario. That would mean that the
devices does not need to be connected to the internet creating an
opportunity to make an ad-hoc interactive table anywhere.

