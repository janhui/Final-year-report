\chapter{Device Camera}

\label{ch:devcamera}
\section{Introduction}

The biggest obstacle currently faced by HuddleLamp project is the antiquation of the Creative Senz3d camera and the Intel Perpetual SDK. This means that at the time of writing this report, it is difficult to contribute towards the HuddleLamp project. Also the practical application of the product has been hindered substantially. In this section I will look at using an Android device as an alternative to the camera currently used by the HuddleLamp group. I will attempt t

\todo[inline] {finish}

\section{Design Choices} \label{design_choice_android_camera}
In this section I will outline some of the main choices, I faced with during this implementation of my project. I will attempt to outline a balanced view of the options I had and justify my decisions.

\subsection{Vision Libraries}
One of the biggest part of the HuddleLamp project is the vision calculations that the computer does. The HuddleLamp project currently uses Opencv as its vision library for different functions such as, calculating the location of the hand of a user and also the locaiton of the mobile devices on the table. In this section I will look at the different vision libraries there are for an Android application.

\subsubsection{OpenCV}
The OpenCV (Open Source Computer Vision) is a library the was created to accelerate the adoption of computer vision in application. 
\subsubsection{JavaCV}
\todo[inline] {not sure about java cv mayb fast cv}
\subsection{Graphics Manipulation} 
\subsubsection{Canny edge Detection}
\subsubsection{Contours}

\subsection{Data Storage}
\subsubsection{Firebase}

Firebase\cite{firebase} is a relatively new back-end as a service
company recently acquired by Google. They are a cloud service provider
from San Francisco who caters a number of products for software developers
building mobile or web application. The service we are more interested
is their Real-time Database service. It provides developers with an
API that which enables data to be synchronised across clients automatically
and stored on the Firebase cloud. The main advantage is the fact that
the data gets synchronised within milliseconds. It also supports a
multitude of platforms such as Android, iOS, JavaScript etc.

It also only requires client-side code meaning the developers has
a relative easy job to create the back-end they required with. Storing
the information as JSON with every information accessible through
their own URL means the data is easy to store and retrieve and there
is also a readily available rest endpoint. 

By providing a simple security management system, Firebase ensures
that the data moved through the apps are encrypted and safe. By ensuring
that the security measures are enforced across all parts of the system,
It helps the developer in eradicating glaring security risks. Firebase
built with scaling and performance in mind. When the data changes
Firebase calculates the minimum number of changes required to keep
all the clients in sync. It also scales linearly depending on the
amount of data being synchronised, decreasing the likely hood of a
developer error. 

The potential of Firebase is enormous. The creators of Firebase hope
that it can serve the needs of 95\% of web service\cite{firebase-wired},
meaning it will be an ideal tool in the stack of our system. One of
its glaring problems is the it is not ideal for processing images,
however the real-time data synchronisation aspect of it make it a
really good candidate for the HuddleTable.


\subsubsection{Meteor}

Meteor\cite{meteor} is a real-time JavaScript framework to come from
the Y-Combinator incubator in 2011. It is primarily used by developers
due to its ability in rapid-prototyping, hence for a MEng project
stack, meteor is an ideal framework to use. It produces cross-platform
code such that for a multiple mobile devices project like HuddleTable
it would cut down on the work required to get most of the devices
to work. It is tightly coupled with the MongoDB, a prominent No-SQL
database. It has an in built publish\textendash subscribe pattern\cite{pub-sub-pattern}
which helps to propagate data changes to clients in real-time without
the need for synchronisation protocols.

Some of the design principles used by Meteor makes it a really good
choice for HuddleTable. It has a Data on the wire motivation meaning,
it only sends the minimum data necessary to re-render the portion
of the page that has changed decreasing the latency. It has a full
stack reactivity\cite{meteor-wiki} meaning that everything gets updated
together when required encouraging its simplicity motivation. Its
created in such a way that it is easy to learn and implement even
for beginners. 

Meteor is the framework of choice used the by HuddleLamp research.
They developed an Object storage on top of Meteor so that all manipulations
on one device become instantly visible on all the other devices. This
allowed the cross platform collaborative work to be done with ease.
Having to only work with one language, JavaScript, on the whole project
means a lot less work for the developer, however the complex algorithm
required for HuddleTable would not be enticing to do in JavaScript.


\subsubsection{AirDrop / Android Beam}

AirDrop\cite{airdrop} is another data sharing service started in
2011 by Apple. It lets you transfer files, links and other content
over the Wi-Fi and Bluetooth between Apple devices. It is a technology
that has been widely appreciated by Apple users and highly congratulated
on. The simple interface to send files back and forth without any
``handshake'' like system and having no restrictions in size of
the files transferred means its a highly appreciated technology which
does not require Internet for it to work. It also has built in function
to control the recipients of the information sent. AirDrop creates
a peer-to-peer Wi-Fi network between the devices and the files are
sent encrypted which means the file transfer is also secure.

Android Beam\cite{android-beam} is a similar technology developed
by Android for their devices. It allows android devices to share content
with each other by pressing the devices back to back. Android uses
the Near field Communication (NFC) hardware to transfer the information
between the devices. However compared to the Apple's AirDrop, Android
beam is slower due to it using NFC instead of Wi-Fi. NFC is also not
available on all Android devices.

These technologies are restricted because they would only work with
each other. By creating a protocol similar to this which would work
across platform would be an ideal scenario. That would mean that the
devices does not need to be connected to the Internet creating an
opportunity to make an ad-hoc interactive table anywhere.

