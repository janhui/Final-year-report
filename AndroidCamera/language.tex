\subsection{Language choice}


\subsubsection{Java} \label{java}

Java is one of the most widely used language for development. It also
has the unique advantage that Java programs compile to a Java bytecode,
which runs on a Java Virtual Machine. This means that as a developer
they would only usually have to create the program once and it would
run on most machines. 

Java also happens to be the native language for Android, making it
an ideal choice of language for an Android developer. Android platform
has a market share of around 81.5\% of mobile devices\cite{market-share}, which means
creating a library for android made the perfect sense. More people
are likely to have an Android device during a meeting and also there
is a more likelihood that there would be a lot more spare Android devices
which can be used to create the screen for the table. Android devices
are also extremely cheap which means even if the users had to buy
some devices to use the interactive table it would be financially viable\cite{cheap-android}.

By creating the library in Java there would be more possibility for
adoption by other users due to the ease in which someone could pick
it up and also due to the large number of Android developers already out there. 


One of the biggest disadvantage of using Android devices with BLE
technology is the the Android Bluetooth Crash bug\cite{bluetooth-share}.
There is a bug in Bluetooth stack of Android where an Android device
can only see 1,990 different Bluetooth MAC addresses, over its lifetime, before the Android
BluetoothService crashes. Android also periodically saves its list
of recent mac address which means this stored list could get filled
up pretty quickly. Small iBeacons such as the Gimbal by Qualcomm generates
a new MAC address for itself every 0.8 seconds which means this list
could get filled very quickly. When is this list gets filled up an
error is shown. The only real solution to this
problem is downloading an application called Bluetooth Crash Resolver by radius
network which is not a perfect solution but the best there is right
now.


\subsubsection{Object C/Swift}

Object C/ Swift is the native language for iOS platform used by Apple
devices. With a 14.8\% market share iOS is the second biggest mobile
platform\cite{market-share}. They recently brought out
another language called Swift which could be used for development in the iOS platform. iOS was a platform
that was brought in 2007 and over the last 8 years, over one billion
iOS devices has been sold\cite{oneBios}.

Over 1.4 million iOS applications have been published in Apple's App
store, showing that large number of developers have adopted Apple products
and are creating apps for the platform. Apple is also the company
that coined the term iBeacon which are a ``new class of low-powered,
low-cost transmitters that can notify nearby iOS 7 or 8 devices of
their presence.'' This means that the iOS platform has a more mature
codebase for BLE technology and also there is a lot more support
for it. The iOS CoreLocation\cite{corelocation} framework is a relatively
more accurate framework, giving iOS a very good advantage when working
with BLE technology. The Bluetooth stack used by Apple products is
a more efficient one compared to the Android Bluetooth stack.

However for a developer there are quite a lot of requirements before being
able to develop for an Apple platform. Unlike Android development
an iOS developer needs to have an Apple computer and an Apple developer
License before they can create apps for the platform, neither of which
are cheap. Apple devices are also rather expensive to buy and it is unlikely for people to have it just laying around.


\subsubsection{JavaScript}

JavaScript is also another viable language for creating applications
for mobile devices especially since the introduction of frameworks
such as PhoneGap\cite{phonegap} and Ionic\cite{ionic}. Mobile Frameworks
such as these work on top of Apache Cordova which allows developers to
create an application using web technologies such as HTML5, CSS3 and
JavaScript instead of using the platform specific APIs like those
in iOS and Android. Cordova wraps these web application to work on
different devices. These applications are called hybrid applications
since they are not truly native mobile or web application.

The use of web-based technologies makes many PhoneGap applications run slower than native applications with similar functionality.
However PhoneGap supports development for iOS, Blackberry, Android,
LG webOS, Microsoft Windows Phone (7 and 8), Nokia Symbian OS, Tizen
(SDK 2.x), Bada, Firefox OS, and Ubuntu Touch making it one of the
best languages to use while creating an application to get the most
exposure for the library. It also supports most of the features of
a phone that a native language supports such as Accelerometer, Compass
etc.



