
\chapter{Conclusion and Future Work}

\label{ch:conclusions}

\section{Applications}


\emph{The project's conclusions should list the things which have been learnt as a result of the work you have done. For example, "The use of overloading in C++ provides a very elegant mechanism for transparent parallelisation of sequential programs", or "The overheads of linear-time n-body algorithms makes them computationally less efficient than O(n log n) algorithms for systems with less than 100000 particles". Avoid tedious personal reflections like "I learned a lot about C++ programming...", or "Simulating colliding galaxies can be real fun...". It is common to finish the report by listing ways in which the project can be taken further. This might, for example, be a plan for doing the project better if you had a chance to do it again, turning the project deliverables into a more polished end product, or extending the project into a programme for an MPhil or PhD.}

\begin{itemize}
\item talk about sucesss
\item talk about bluetooth failure
\item doesnt do spatial menus
\item talk about how  I might start differently (microservice) pros and cons
\item nearables
\end{list}


\section{Future Work}
\begin{itemize}
\item port it to microservice
\item using a more resourceful tablet
\item decrease jni calls
\item port it to more mobile devices
\end{itemize}