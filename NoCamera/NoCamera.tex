\chapter{No Camera}

\label{ch:no_camera}

\section{Design Choices}

\subsection{Radio Frequency Signal}

\subsubsection{Wi-Fi}

Wi-Fi is one of the most used Radio Frequency (RF) signals out there.
It is used for computer networking using 2.4 GHz UHF and 5 GHz SHF
ISM radio bands\cite{wifi-wikipedia}. The services provided are also
numerous, from consulting web pages to watching on demand video sequences. 

One of the biggest advantages of Wi-Fi is the fact that it\textquoteright s
widely used. Almost all location has Wi-Fi now, which means the setup
cost and maintenance cost of Wi-Fi would be cheap. It also has the
ability to transfer complicated data between its Access point (AP)
and the mobile devices, making it highly adaptable. 

Wi-Fi has been used for indoor positioning before by various research
groups. Two positioning techniques have been used for this research
called Signal Strength Cartography and Wave propagation. Signal Strength
Cartography is a reference-based system where you map the area offline
with coordinates and the strength of the Wi-Fi signal. Then when you
are trying to locate a device use the coordinates and the strength
to work out where exactly the device is. There are two main steps
for this mapping, an offline mapping step, which could be done either
by simulation or manually using measurements, and an online positioning
step, which could be done either using a probabilistic method or deterministic
method. 

Wave propagation approach is a mathematical attempt at finding the
distance between the device and the transmitter by finding a relation
between the distance and the signal strength. By using at least 3
AP we can use wave propagation and trilateration to work out the position
of the device. 

However there are various disadvantages of using Wi-Fi to work out
positioning of the devices. There are usually only one AP per location,
however we would require more than one, to get more accurate positioning
of the devices. Wi-Fi works mainly by sending signals in channels,
so having more than one Wi-Fi AP in a location could produce congestion
in channels, which would create interference. Wi-Fi also has a higher
power consumption. This is because of the larger range in which Wi-Fi
sends out its signal. 

There is hope for using Wi-Fi as the main Radio frequency for our
product in the future after the Wi-Fi Standards agency brings out
Wi-Fi -aware. Which is proximity based discovery system. Introduction
of this system could mean that you don\textquoteright t require any
extra AP, and the devices could interact with each other without any
outside help. 


\subsubsection{Global Positioning System (GPS)}

GPS is one of the most widely used features in our devices. Over the
past few years the use of GPS has increased significantly due to the
increasing social network activity. Nowadays almost every app that
we use through our mobile devices request for GPS access which shows
how much it has come forward. GPS is a space based navigation system
using the satellites. The Department of Defence of USA, using the
24 satellites already in orbit, initially introduced it. 

GPS satellites transmit data continuously, which contains their current
time and position. A GPS receiver listens to multiple satellites and
solves equations to determine the exact position of the receiver and
its deviation from true time. At a minimum, four satellites must be
in view of the receiver in order to compute four unknown quantities
(three position coordinates and clock deviation from satellite time). 

The main advantage of GPS is that it\textquoteright s widely available.
Almost all devices have a GPS receiver on it. It also has a very large
range. However it is not a viable option for us because of the fact
that the signal comes from outer space and our devices are usually
indoors. GPS like any radio frequency signal is prone to absorption
and diffraction by the roof and walls in the building. GPS also has
high power consumption. GPS tends to be not very accurate, averaging
\textasciitilde{}10m which would make it not feasible for our use. 


\subsubsection{Bluetooth}

Classic Bluetooth, usually known as Bluetooth, was the codename for
a project by Special Interest Group (SIG), collaboration between major
companies, like Ericsson, Intel, and Nokia. Bluetooth was invented
for short-range wireless communication with devices. Bluetooth uses
radio signals in the 2.4 GHz range to transmit data. This range is
globally license free range; therefore there is no extra cost for
deployment. Bluetooth is not targeted any specific application, therefore
a multitude of applications has used Bluetooth in various different
ways from transferring files to streaming songs.

SIG is in charge of the specification of Bluetooth and creating a
road map for it going forward. To standardise the form of communication
through Bluetooth, SIG has defined a set of profiles that needs to
be made available by the manufacturer. Devices usually only have a
subset of these profiles enabled. Some of these profiles are: 
\begin{itemize}
\item A2DP Advanced Audio Distribution Profile. Used for Streaming audio. 
\item HFP Hands-Free Profile. Used for hands free kit to make calls in cars 
\item HID Human Interface Device Profile. Provides support for input devices
like keyboards, mice and game controllers.
\end{itemize}
Bluetooth is aimed at applications that require short-range communication
typically few meters. The actual range an application make depends
on the Bluetooth Class the application has and also external circumstances
like absorption, diffraction etc. Bluetooth devices are separated
out into 3 different classes with separate transmission power. The
full picture is shown in Table 1.

\begin{tabular}{|c|c|c|c|}
\hline 
Class & Max Power Output & Max Range & Power Level Control\tabularnewline
\hline 
\hline 
1 & 100 mW & 100 m  & Mandatory\tabularnewline
\hline 
2 & 2.5 mW & 10 m & Optional\tabularnewline
\hline 
3 & 1 mW & 1 m & Optional\tabularnewline
\hline 
\end{tabular}
\begin{table}[H]
\protect\caption{Bluetooth Power class}
\end{table}


Due to the different profiles and more than sufficient range and relatively
low power consumption, Bluetooth shows considerable potential to be
used as our Radio frequency signal for our positioning. However they\textquoteright re
some aspects of it, which makes it difficult for Bluetooth to be used.
The main problem is the security constraints of Bluetooth, which requires
the devices to go through a pairing process with each other. 

In Bluetooth technology a device could take either the master role
or the slave role. Before the devices can communicate with each other
they have to discover each other and specify which profile they are
going to use. Pairing is done by: 
\begin{enumerate}
\item Master devices continuously broadcast \textquotedblleft inquiry messages\textquotedblright{}
\item These messages will be picked up by nearby devices that are \textquotedblleft discoverable\textquotedblright{} 
\item These devices will respond with a message containing their name, profiles
they support and other technical details. 
\item Using these details the master device can establish a connection with
the correct profile 
\end{enumerate}

\subsubsection{Bluetooth Low Energy (BLE)}

BLE is a relatively new technology, which came out in 2010 with the
new specification for Classic Bluetooth 4.0. BLE was introduced for
devices that are predominantly used for monitoring and control, like
sensors values and control commands, where there is no need for large
amount of data transfer.


\paragraph*{Modes\protect \\
}

One of the biggest improvements introduced for BLE that is not there
for Bluetooth is the addition of the new mode. This new mode takes
away the necessity for pairing to exchange data. In this \textquotedblleft broadcast\textquotedblright{}
mode the device can send data in the advertisement channel. There
are 4 different modes in BLE: 
\begin{description}
\item [{Central}] : Similar to the Bluetooth master role, can have multiple
connections. 
\item [{Peripheral:}] A device can only have one active connection with
the central mode 
\item [{Broadcaster}] : Where you send data in the advertisement channel
\item [{Observer:}] Where you listen to the advertisement channel
\end{description}

\paragraph*{Scanning\protect \linebreak{}
}

Another big improvement BLE has brought is the ability to discover
other devices in two different modes. BLE enabled devices can now
passively and actively scan for connectable devices.
\begin{itemize}
\item Passive scanning - a central device listens to the advertising channel
passively to capture all the packets transmitted by connectable devices 
\item Active scanning - a central device listen for advertisement packets
and when it receives the packet, it checks whether the sender is connectable
through looking at the mode. If it is then a scan request packet is
send to gain more information.
\end{itemize}
Devices may advertise as seldom as once every 10 seconds or as fast
as every 20 millisecond. 


\paragraph*{Range\protect \\
}

Similar to the Classic Bluetooth, the range of BLE is determined by
the transmitting power and the interference that it might experience.
BLE has a transmitting power up +10dBm, which gives it a range of
300m theoretically. However BLE usually use a power of 0dB or less
which gives it a range of about 50m. Even though BLE is meant to have
low-energy consumption, it has a bigger range than Classic Bluetooth
under maximum power due to smaller packet size. 


\subsubsection{Others eg Zigbee}

There are various other technology out there that could be used such
as Zigbee. However since these are not commonly supported by mobile
devices. Therefore they have not been considered for the implementation.




