\section{Evaluation}
I devised a few methods to test my implementation on it accuracy. Since there were multiple facet, I have separate tests for each of them.
\subsection{Distance}
The distance between the beacon and the device is one of the biggest contributor towards the trilateration part of the positioning algorithm. The test I devised was to log the values that the distance function showed for every 5cm interval between 5 cm - 50cm.

I tested the distance given by the function computeAccuracy() first. Table \ref{tableOfComputeAccuracyOldAndroid} shows the average distance shown by the library at each intervals. However as you can see from the figure () there was a large error rate in the values. 

My research had shown that another potential method of getting the distance of a receiver from the BLE emitter was by analysing the RSSI signal and the Transmission power. Therefore I logged the RSSI value and Transmission Power for those distances to calculate the relationship between RSSI, Transmission Power and the actual distance. Table \ref{tableOfRSSIOldAndroid} shows the result for each distance. By plotting a graph on it, I was able to calculate the relationship between them. Figure \ref{oldRSSIDistance} shows the graph, which shows that the distance is \enph{y = mx+c}.
\todo[inline]{do the equation}
However even using the formula for distance I found out that the values were still error prone. This meant that the trilateration will not work, therefore the underlying basis of my project was flawed.
\subsubsection{Problems}
One of the big problems face during this test was the Bluetooth Share Bug\cite{bluetooth-share} which kept showing up on one of the tablet used. Another was the face there seemed to be a discernible difference in reading between different versions of Android OS. Therefore I also did these tests in Android versions () () . The results of which could be seen in Table \ref{tableOfRssiNewAndroid} and \ref{tableOfComputeAccuracyNewAndroid} and Figure \ref{newRSSIDistance} and \ref{newComputeAccuracyDistance}.

\todo [inline]{find android versions}
\subsection{Displacement}
Another major facet of my project was the sensor fusion idea to get the positioning correct. I decided to conduct a test for this by moving the tablet across a distance of 5 - 50 cm in every 5 cm intervals and calculating the displacement. Table() shows the results for it.

\begin{figure}
\centering
\fbox{
\begin{tikzpicture}
\begin{axis}[
    enlargelimits=false,    
    title={},
    xlabel={Actual Distance (m)},
    ylabel={Calculated Distance (m)},
    legend pos=north west,
    xmin=0
]
\addplot table [only marks, mark = *] {\distanceOldRSSI};
\addplot [thick, red] table[y={create col/linear regression={y=X}}]{\distanceOldRSSI};
\addlegendentry{%
$\pgfmathprintnumber{\pgfplotstableregressiona} \cdot x
\pgfmathprintnumber[print sign]{\pgfplotstableregressionb}$}
\end{axis}
\end{tikzpicture}}
\caption{Graph to show the relationship of (RSSI/Transmission Power) and Distance old android version}
\label{oldRSSIDistance}
\end{figure}

\begin{figure}
\centering
\fbox{
\begin{tikzpicture}
\begin{axis}[
    enlargelimits=false,    
    title={},
    xlabel={Actual Distance (m)},
    ylabel={Calculated Distance (m)},
    legend pos=north west,
    xmin=0
]
\addplot table [only marks, mark = *] {\distanceOldComputeAccuracy};
\addplot [thick, red] table[y={create col/linear regression={y=X}}]{\distanceOldComputeAccuracy};
\addlegendentry{%
$\pgfmathprintnumber{\pgfplotstableregressiona} \cdot x
\pgfmathprintnumber[print sign]{\pgfplotstableregressionb}$}
\end{axis}
\end{tikzpicture}}
\caption{Graph to show the relationship of computeAccuracy and Distance for old android version}
\label{oldComputeAccuracyDistance}
\end{figure}

\begin{figure}
\centering
\fbox{
\begin{tikzpicture}
\begin{axis}[
    enlargelimits=false,    
    title={},
    xlabel={Actual Distance (m)},
    ylabel={Calculated Distance (m)},
    legend pos=north west,
    xmin=0
]
\addplot table [only marks, mark = *] {\distanceNewComputeAccuracy};
\addplot [thick, red] table[y={create col/linear regression={y=X}}]{\distanceNewComputeAccuracy};
\addlegendentry{%
$\pgfmathprintnumber{\pgfplotstableregressiona} \cdot x
\pgfmathprintnumber[print sign]{\pgfplotstableregressionb}$}
\end{axis}
\end{tikzpicture}}
\caption{Graph to show the relationship of computeAccuracy and Distance for newer Android version}
\label{newRSSIDistance}
\end{figure}

\begin{figure}
\centering
\fbox{
\begin{tikzpicture}
\begin{axis}[
    enlargelimits=false,    
    title={},
    xlabel={Actual Distance (m)},
    ylabel={Calculated Distance (m)},
    legend pos=north west,
    xmin=0
]
\addplot table [only marks, mark = *] {\distanceNewRSSI};
\addplot [thick, green] table[y={create col/linear regression={y=X}}]{\distanceNewRSSI};
    \addlegendentry{%
        $\pgfmathprintnumber{\pgfplotstableregressiona} \cdot x
        \pgfmathprintnumber[print sign]{\pgfplotstableregressionb}$} %
\end{axis}
\end{tikzpicture}}
\caption{Graph to show the relationship of (RSSI/Transmission Power) and Distance for newer Android version}
\label{newComputeAccuracyDistance}
\end{figure}


\subsubsection{Problems}

\subsection{Trilateration}

\subsection{Strengths and Weakness}
 