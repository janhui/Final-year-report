\section{Evaluation}
I devised a few methods to test my implementation on it accuracy. Since there were multiple facet, I have separate tests for each of them.
\subsection{Distance}
The distance between the beacon and the device is one of the biggest contributor towards the trilateration part of the positioning algorithm. The test I devised was to log the values that the distance function showed for every 5cm interval between 5 cm - 50cm.

I tested the distance given by the function computeAccuracy() first. Table () shows the average distance shown by the library at each intervals. However as you can see from the figure () there was a large error rate in the values. 

My research had shown that another potential method of getting the distance of a receiver from the BLE emitter was by analysing the RSSI signal and the Transmission power. Therefore I logged the RSSI value and Transmission Power for those distances to calculate the relationship between RSSI, Transmission Power and the actual distance. Table () shows the result for each distance. By plotting a graph on it, I was able to calculate the relationship between them. Figure () shows the graph. Figure () shows that the distance is y = mx+c.

However even using the formula for distance I found out that the values were still error prone. This meant that the trilateration will not work, therefore the underlying basis of my project was flawed.
\subsubsection{Problems}
One of the big problems face during this test was the Bluetooth Share Bug\cite{bluetooth-share} which kept showing up on one of the tablet used. Another was the face there seemed to be a discernible difference in reading between different versions of Android OS. Therefore I also did these tests in Android versions () () . The results of which could be seen in Table () () and Figure ().

\todo [inline]{find android versions}
\subsection{Displacement}
Another major facet of my project was the sensor fusion idea to get the positioning correct. I decided to conduct a test for this by moving the tablet across a distance of 5 - 50 cm in every 5 cm intervals and calculating the displacement. Table() shows the results for it.
\subsubsection{Problems}

\subsection{Trilateration}

\subsection{Strengths and Weakness}
 